The current exposition is based on Digne--Michel~\cite{DiMBook} and~\cite{DiMDel}. Many of the ideas are already present in Brou\'e--Michel~\cite{BrMi}. Details and more complete information on reductive groups can be found for instance in the books of Borel~\cite{borel}, Humphreys~\cite{humphreys}, or Springer~\cite{springer}. For their representation theory, see the books of Carter~\cite{carter} or Digne--Michel~\cite{DiMBook}. 

The construction of cuspidal characters was first done  for $\GL_n$ by Green (1955) using a complicated inductive construction. Macdonald(1970) predicted that the cuspidal characters should be associated to some twisted tori, that is, some $\bT_w$ with $w\ne 1$. Finally, Deligne and Lusztig (1976) constructed a variety associated to twisted tori such that the cuspidal representations occur in its $\ell$-adic cohomology.

What we have looked at is a simplified version of  a particular case of the
Brou\'e conjecture. The general Brou\'e conjecture is about the cohomology of a nonnecessarily constant sheaf on the variety $\bX_w$ where moreover the base ring is the ring of $\ell$-adic integers instead of $\Qlbar$. 

Proposition \ref{P:DLDeligneThm}
was proved by Deligne in the paper \cite{Deligne}. This paper was
motivated by a construction of Bondal
and by a question of Brou\'e and Michel when they were writing the paper
\cite{BrMi} and wanted to define Deligne-Lusztig varieties associated to all
elements of an Artin-Tits monoid (see \cite[1.6]{BrMi}).
Another proof was given recently by Gaussent, Guiraud and Malbos
(arXiv:1203.5358).

The generalization (Proposition \ref{P:DLBicategory})
we give in the Appendix appeared in \cite{DiMPar}.

%%%%%%%%%%%%%%%%%%%%%%%%%
% END OF CHAPTER DL
